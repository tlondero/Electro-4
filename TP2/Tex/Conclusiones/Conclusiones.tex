
Se obtuvo conocimiento del funcionamiento del IC SG3525, tanto de su funcionamiento interno como de sus funcionalidades. Además, se obtuvieron conocimientos prácticos en el armado de un circuito que permite variar tanto la frecuencia como el duty del generador de pulsos. 

Se pudo simular en LTSpice el circuito y se calculó un snubber para este, midiendo su eficiencia (variando tanto carga como tensión de salida) y comparando tanto los valores teóricos con simulados. También se modeló la función transferencia de la fuente Flyback mediante el uso de la promediación de variables de estados. Obteniendo la función transferencia, una particularidad de la topología Flyback es la existencia de un cero en el semiplano derecho. Lo cual trae el problema de que este cero querrá traer los polos al semiplano derecho, asi provocando inestabilidad.\\
Para asegurar estabilidad se diseño un compensador  para ubicar los polos en un sector que asegure estabilidad del sistema, asi cerrando el lazo de realimentación en una configuración no inversora. Obteniendo la ganancia de lazo, se obtuvo su diagrama de Bode de alli el margen de fase, que es un valor clave al analizar la estabilidad del sistema. al igual que el diagrama de polos y ceros.

Finalmente se implemento en una placa multiperforada teniendo en cuenta los lineamientos para el diseño de circuitos de potencia provistos por la cátedra. Agregando un capacitor en la entrada. al igual que múltiples capacitores en la carga para auementar la capacidad y bajar la ESR equivalente.
Haciendo pistas únicamente de cables y no de estaño.
 
En resumen realizó el diseño de un convertidor DC-DC topología Flyback realimentado exitosamente, con 2 salidas simétricas y la posibilidad de variar la tensión entre un rango de $0.8 \ V$ a $3 \ V$, con una tensión de referencia (la cual podría ser reemplazada por un potenciómetro conectado entre la tensión de referencia del SG3525, el pin no inversor y masa).
Se vió el efecto de la ESR sobre la salida y como mitigarlo. Además se observó el comportamiento de la fuente al variar los valores de carga.
