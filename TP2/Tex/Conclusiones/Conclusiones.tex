Se realizó el diseño de un convertidor DC-DC topología Flyback exitosamente, con 2 salidas simétricas y la posibilidad de variar la tensión entre un rango de $0.8 \ V$ a $3 \ V$.

Se obtuvo conocimiento del funcionamiento del IC SG3525, tanto de su funcionamiento interno como de sus funcionalidades. Además, se obtuvieron conocimientos practicos en el armado de un circuito que permite variar tanto la frecuencia como el duty del generador de pulsos. 

Se pudo simular en LTSpice el circuito y se calculó un snubber para este, midiendo su eficiencia y comparando tanto los valores teóricos con simulados. También se modeló la función transferencia de la fuente Flyback mediante el uso de la promediación de variable de estados. Obteniendo la función transferencia, una particularidad de la topología Flyback es la existencia de un cero en el semiplano derecho.\\
 Luego, se diseño un compensador para ubicar los polos en un sector que asegure estabilidad del sistema. Finalmente se implemento en una placa multiperforada. 