Se realizó el diseño de un convertidor DC-DC topoogía Flyback exitosamente, con 2 salidas simétricas y la posibilidad de variar la tensión entre un rango de [0.8,3] V.\\
Se obtuvo conocimiento del funcionamiento del IC SG3525, tanto teórico de su funcionamiento interno y sus funcionalidades. Al igual que practico en el armado de un circuito que permite variar tanto la frecuencia como el duty del generador de pulsos. 
 Se pudo simular en LTSpice le circuito y se calculó un snubber para el circuito, midiendo su eficiencie y coparando tanto teorico con siulado. También se modeló la función transferencia de la Flyback mediante el uso de la promediación de variable de estados. Al igual que el diseño de un compensador para ubicar los polos en un sector que asegure estabilidad del sistema. Finalmente se implemento en una placa multiperforada. 