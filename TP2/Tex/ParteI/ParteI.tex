%\input{../Informe/Header.tex}
%
%\begin{document}

\subsection{SG3525A}

\subsubsection{Debajo de la tensión de operación}

Este dispositivo cuenta con un pin de control llamado \quotes{Shutdown}. Este pin controla tanto el circuito de Soft-Start como las etapas de salida, proveyendo apagadas automáticas a través de pulsos de shutdown.

Al haber una tensión inferior al mínimo ($8 \ V$) en la entrada, este sistema de shutdown se activa, inhabilitando las salidas y los cambios en el capacitor de Soft-Start.

\subsubsection{Señal a la salida}

Para seleccionar la frecuencia de la señal a la salida del integrado, de deben conectar dos resistencias y un capacitor en los pines $C_T$ y $R_T$. El criterio de selección viene dado por las siguientes limitaciones:

\begin{equation}
	f_s = \frac{1}{C_T \left( 0.7 R_T + 3 R_D \right)}
\end{equation}

\begin{equation*}
\begin{gathered}
2.0 \ k\Omega \leqslant R_T \leqslant 150 \ k\Omega \\
0 \ \Omega \leqslant R_D \leqslant 550 \ \Omega \\
1 \ nf \leqslant C_T \leqslant 200 \ nf \\
\end{gathered}
\end{equation*}

Para conseguir una frecuencia de $100 \ kHz$ basta con tomar:
\begin{equation}
\begin{gathered}
R_T = 10 \ k\Omega \\
R_D = 0 \ \Omega \\
C_T = 1.43 \ nf \\
\end{gathered}
\end{equation}

Para conseguir un Duty deseado, basta con colocar una tensión de referencia en el inversor (pin 1).

\subsubsection{Soft-Start}

El pin de Soft-Start cumple con la función de limitar el Duty cicle al principio, hasta que el capacitor de SS esté cargado. %Una vez que esto ocurre (...)

Cuando se está empleando una fuente realimentada, si al principio la salida de dicho circuito es nula, se busca que este no produzca un Duty tal para llegue a la tensión deseada. De esta forma se evita que se sobredimensione y se pase del valor que se necesita, disminuyendo así las oscilaciones iniciales.

%Limita el duty al principio hasta que el capacitor de SS esté cargado -> pin 9 (COM) va al comparador PWM, dependiendo esa tensión tenemos el duty.
%SS -> Cuando tengo una fuente realimentada, que a la salida es 0 al principio, el circuito no produzca un duty tal para que llegue  a la tensión que quiere y se pasa.

\subsubsection{Shutdown}

Es posible implementar un circuito limitador de corriente utilizando el pin de Shutdown con una resistencia de shunt y un BJT. Se mide la corriente hasta que esta sea mayor a la deseada. Cuando esto se de, el BJT activa el pin apagando la salida del circuito.

%\end{document}