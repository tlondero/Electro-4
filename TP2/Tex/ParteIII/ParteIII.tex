%\input{../Informe/Header.tex}
%
%\begin{document}

\subsection{Matrices de estado}

Para obtener teóricamente la transferencia del circuito, se vale del método de variables de estado se llega a las siguientes matrices:

\begin{multicols}{2}
\begin{equation}
\mathbb{A} = 
\begin{pmatrix}
-\frac{DR_1}{L_1} + \frac{(1 - D) n^2 R_C R}{(R - R_C) L_1} & \frac{(1-D) n R}{(R - R_C) L_1}\\
-\frac{(1 - D) n R}{(R - R_C) C} & -\frac{D }{(R - R_C) C} - \frac{(1 - D)}{(R - R_C) C}
\end{pmatrix}
\end{equation}

\begin{equation}
\mathbb{B} = 
\begin{pmatrix}
	-\frac{D}{L_1} & 0\\
	0 & 0
\end{pmatrix}
\end{equation}

\begin{equation}
\mathbb{C} = 
\begin{pmatrix}
	-\frac{(1-D)n R R_C}{R - R_C} & \frac{D R}{R + R_C} + \frac{(1-D) R}{R - R_C}
\end{pmatrix}
\end{equation}

\begin{equation}
\mathbb{D} = 
\begin{pmatrix}
	0 & 0
\end{pmatrix}
\end{equation}
\end{multicols}

Se toman los valores seleccionados en la Sección (\ref{sec:parteii}). Además se consideran las resistencias tanto de $N_1$ y de los capacitores de salida (para las cuentas se consideran los dos capacitores en paralelo como uno solo con una única ESR) siendo estos $R_L = 0.001 \ \Omega$ y $R_C = 0.001 \ \Omega$ respectivamente. De esta forma se obtiene la transferencia del sistema:
\begin{equation}
	G(s) = \mathbb{C} \left(s \mathbb{I} - \mathbb{A} \right)^{-1} \mathbb{B} = \frac{-15.75 s + 3.351 \cdot 10^{8}}{s^2 + 3002s + 2.346 \cdot 10^{9}}
\end{equation}

\subsection{Compensador}

Se utiliza el siguiente circuito como compensador:
\begin{figure}[H]
	\centering
	\includegraphics[width=0.3\linewidth, page = 1]{ImagenesParteIII/CircuitsP3.pdf}
	\label{fig:compensador}
	\caption{Circuito compensador del sistema.}
\end{figure}

La transferencia de este sistema es la siguiente:
\begin{equation}
	H(s) = \frac{ 1 }{ R_{f1} C_{C2} } \cdot \frac{s + \frac{1}{ R_{C1} C_{C1}}}{ s \left( s + \frac{R_{C1} + R_{C2}}{R_{C1} C_{C1} C_{C2}} \right)}
\end{equation}

Se emplean los siguientes valores:
\begin{multicols}{2}
\begin{itemize}
	\item $R_{C1} = 10 \ k\Omega$
	\item $R_{f1} = R_{f2} = 1 \ k\Omega$
	\item $C_{C1} = 10 \ nf$
	\item $C_{C2} = 1 \ \mu f$
	\item $N_2 = 1 \ \mu H$
\end{itemize}
\end{multicols}

Con esos valores, el compensador queda:
\begin{equation}
	H(s) = \frac{0.0001 s + 1}{ 1 \cdot 10^{-7} s^2 + 0.00101 s }
\end{equation}

Con el sistema realimentado, se grafican los diagramas de Bode y el plano Z del sistema.
\begin{figure}[H]
	\centering
	\includegraphics[width=0.9\linewidth]{ImagenesParteIII/Bode.png}
	\label{fig:bode}
	\caption{Diagrama de Bode.}
\end{figure}

\begin{figure}[H]
	\centering
	\includegraphics[width=0.9\linewidth]{ImagenesParteIII/Rlocus.png}
	\label{fig:zplane}
	\caption{Plano Z, diagrama de polos y ceros.}
\end{figure}

%\end{document}