\documentclass[border={0.5cm 0.5cm 0cm 0.5cm}, 11pt, tikz, multi=page]{standalone}
\usepackage[utf8]{inputenc}
\usepackage[spanish, es-tabla, es-noshorthands]{babel}

\usepackage{tikz}
\usepackage{textcomp}
\usetikzlibrary{shapes,arrows}

\usepackage{amsmath}
\usepackage{amsfonts}
\usepackage{amssymb}
\usepackage{float}
\usepackage{graphicx}
\usepackage{caption}
\usepackage{subcaption}
\usepackage{multicol}
\usepackage{multirow}
\setlength{\doublerulesep}{\arrayrulewidth}
\usepackage{booktabs}
\usepackage{pgfplots}

\newcommand{\quotes}[1]{``#1''}
\usepackage{array}
\newcolumntype{C}[1]{>{\centering\let\newline\\\arraybackslash\hspace{0pt}}m{#1}}
\usepackage[american]{circuitikz}
\usepackage{fancyhdr}
\usepackage{units}

% Definition of blocks:
\tikzset{%
  block/.style    = {draw, thick, rectangle, minimum height = 3em,
    minimum width = 3em},
  sum/.style      = {draw, circle, node distance = 2cm}, % Adder
  input/.style    = {coordinate}, % Input
  output/.style   = {coordinate}, % Output
  >=Stealth
}

% Defining string as labels of certain blocks.
\newcommand{\suma}{\Large $\Sigma$}
\newcommand{\inte}{$\displaystyle \int$}
\newcommand{\derv}{\huge $\frac{d}{dt}$}

\begin{document}

%%%%%%%%%%%%%%%%%%%%%%%%%%%%%%%%%%%%%%%%%%%%%%%%%%%%%%%%%%%%%%%%%%%%%
%							CIRCUITOS								%
%%%%%%%%%%%%%%%%%%%%%%%%%%%%%%%%%%%%%%%%%%%%%%%%%%%%%%%%%%%%%%%%%%%%%

%TYPE 2 COMPENSATOR
\begin{page}
\begin{circuitikz}[american voltages]

	\draw
		(0,0) node[op amp](op){}
		
		(op.out) -- ++(0.25,0) -| node[pos=.5](aux5){} ++ (0,2) node[](aux3){}		 
		(aux5.center) -- ++ (0.5,0) node[ocirc, label=$V_e$](){}
		(aux3.center) ++ (0,2) node[](aux4){}
		
		(op.+) -| ++ (-1,-1) node[ocirc, label=below:$V_{ref}$](){}
		
		(op.-) -- ++ (-2,0) node[](aux1){} to[R, l=$R_{f1}$] ++ (0,2) node[ocirc, label=$V_{out}$](){}		
		(aux1.center) to[R, l_=$R_{f2}$] ++ (0,-2) node[ground](){}
		
		(op.-) ++ (-1,0) -- ++ (0,1.5) node[](aux2){} to[R, l=$R_{C1}$] ++ (2,0) to[C, l=$C_{C1}$] (aux3.center) 
		(aux2.center) -- ++(0,2) to[C, l=$C_{C2}$] (aux4.center) -- (aux3.center)
	;
	
%	\draw[color = red]
%		(aux1.center) node[circ](){1}
%		(aux2.center) node[circ](){2}
%		(aux3.center) node[circ](){3}
%		(aux4.center) node[circ](){4}
%		(aux5.center) node[circ](){5}
%	;

\end{circuitikz}
\end{page}

\end{document}