\input{../Informe/Header.tex}

\begin{document}

\subsection{Introducción}

Para el estudio del modo discontinuo de la fuente estudiada anteriormente, se calculó la corriente media $I_{L_b} = I_{o_b}$ de boundary de la bobina, la cual es la misma que la corriente media de salida. El valor anterior de $\Delta I_L$ fue de $494.404mA$ por lo que la corriente media de boundary será

\begin{equation}
I_{L_b} = \frac{\Delta I_L}{2} = 247.202mA
\label{ej4:eq:il_boundary}
\end{equation}

Por esta razón, si la corriente de salida es menor que $I_{L_b}$, la fuente trabajará en modo discontinuo. Se seleccionó una resistencia de salida de $R_o = 500\Omega > R_{o_{min}} = \frac{V_o}{I_{L_b}} = 97.1\Omega$ para obtener resultados más significantes y se utilizó un duty cycle $D = 0.665$ para conservar los $24V$ de salida requeridos. A continuación se detallan las curvas simuladas.

 

\end{document}