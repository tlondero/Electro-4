\input{../Informe/Header.tex}

\begin{document}

\subsection{Introducción}

Para el estudio del modo discontinuo de la fuente estudiada anteriormente, se calculó la corriente media $I_{L_b} = I_{o_b}$ de boundary de la bobina, la cual es la misma que la corriente media de salida. El valor anterior de $\Delta I_L$ fue de $494.404mA$ por lo que la corriente media de boundary será

\begin{equation}
I_{L_b} = \frac{\Delta I_L}{2} = 247.202mA
\label{ej4:eq:il_boundary}
\end{equation}

Por esta razón, si la corriente de salida es menor que $I_{L_b}$, la fuente trabajará en modo discontinuo. Se seleccionó una resistencia de salida de $R_o = 500\Omega > R_{o_{min}} = \frac{V_o}{I_{L_b}} = 97.1\Omega$ para obtener resultados más significantes y se utilizó un duty cycle $D = 0.665$ para conservar los $24V$ de salida requeridos. A continuación se detallan las curvas simuladas.

\begin{figure}[H]
	\centering
	\begin{minipage}{0.495\textwidth}
		\centering
		\includegraphics[width=\textwidth]{ImagenesEjercicio-4/vgs} % first figure itself
		\caption{Tensión $V_{gs}$ en modo discontinuo.}
		\label{ej4:fig:vgs}
	\end{minipage}\hfill
	\begin{minipage}{0.495\textwidth}
		\centering
		\includegraphics[width=\textwidth]{ImagenesEjercicio-4/ig} % second figure itself
		\caption{Corriente $I_{g}$ en modo discontinuo.}
		\label{ej4:fig:ig}
	\end{minipage}
\end{figure}

\begin{figure}[H]
	\centering
	\begin{minipage}{0.495\textwidth}
		\centering
		\includegraphics[width=\textwidth]{ImagenesEjercicio-4/vds} % first figure itself
		\caption{Tensión $V_{ds}$ en modo discontinuo.}
		\label{ej4:fig:vds}
	\end{minipage}\hfill
	\begin{minipage}{0.495\textwidth}
		\centering
		\includegraphics[width=\textwidth]{ImagenesEjercicio-4/id} % second figure itself
		\caption{Corriente $I_{D}$ en modo discontinuo.}
		\label{ej4:fig:id}
	\end{minipage}
\end{figure} 

\begin{figure}[H]
	\centering
	\includegraphics[width=0.7\linewidth, page=1]{ImagenesEjercicio-4/il}
	\caption{Corriente de la bobina en modo discontinuo.}
	\label{ej4:fig:il}
\end{figure}

Se 

\end{document}