\input{../Informe/Header.tex}

\begin{document}

\subsection{Introducción}

Se analizó la conmutación de un MOSFET \href{https://www.vishay.com/docs/91019/91019.pdf}{IRF530} de potencia en un circuito con carga inductiva, utilizando un diodo \href{https://www.onsemi.com/pdf/datasheet/mur420-d.pdf}{MUR460} de potencia para proporcionar un camino a la corriente durante el apagado del MOSFET y no dañar al circuito.

%Aca puede ir la foto del circuito

Para la conmutación del MOSFET se utilizó un periodo de $T_s = 20 \mu s$ y un duty cycle de $D = 50 \%$.

\subsection{Circuito en la Teoría}

En la teoría, se consideró al diodo MUR460 como ideal excepto por la caída de potencial de la juntura en directa, siendo esta extraída de la datasheet, con un valor de $V_{D_{on}} = 1.3V$. Además, se consideró a la bobina como ideal con resistencia serie nula.

\begin{figure}[H]
	\centering
	\includegraphics[width=0.7\linewidth, page=1]{ImagenesEjercicio-1/CircuitsEj1}
	\caption{Circuito para el estudio de la conmutación del MOSFET.}
	\label{ej1:fig:circuito}
\end{figure}

\subsubsection{Primer Hemicircuito: MOSFET ON}

Cuando el MOSFET se encuentra encendido, se forma un circuito RL entre la bobina y la $R_{ds_{on}}$ del MOSFET. Como el diodo se encuentra con su ánodo conectado a aproximadamente tierra, y su cátodo conectado a $V_i$, este se encuentra en inversa por lo que no circula corriente a través de él.

\begin{figure}[H]
	\centering
	\includegraphics[width=0.4\linewidth, page=3]{ImagenesEjercicio-1/CircuitsEj1}
	\caption{Hemicircuito con MOSFET encendido.}
	\label{ej1:fig:circuito_on}
\end{figure}

Resolviendo el circuito RL, se puede observar en las ecuaciones (\ref{ej1:eq:IL_on}) y \ref{ej1:eq:VL_on} que durante el MOSFET se encuentra encendido aumenta la energía almacenada en la bobina. Sobre el MOSFET caen $V_{ds_{on}} = I_{L_{on}}\cdot R_{ds_{on}}$

\begin{equation}
	 I_{L_{on}}(t) = \left( I_{L_{off}}\left(t= \frac{n}{f_{sw}}\right) -\frac{V_2}{R_{ds_{on}}+R_2}\right) e^{-\frac{R_{ds_{on}}}{L}t} + \frac{V_2}{R_{ds_{on}}} \ \ \ \ n \in \mathbb{N}
\label{ej1:eq:IL_on}
\end{equation}

\begin{equation}
	V_{L_{on}}(t) = \left( V_2 - I_{L_{off}}\left(t= \frac{n}{f_{sw}}\right)\cdot (R_{ds_{on}}+R_2) \right)e^{-\frac{R_{ds_{on}}+R_2}{L}t} \ \ \ \ n \in \mathbb{N}
\label{ej1:eq:VL_on}
\end{equation}

Donde $t=\frac{n}{f_{sw}}$ son los momentos en los que el MOSFET conmuta de apagado a encendido y $I_{L_{off}}\left(t= \frac{n}{f_{sw}}\right)$ la corriente en la bobina en dicho momento.

\subsubsection{Segundo Hemicircuito: MOSFET OFF}

Una vez apagado el MOSFET, la bobina posee la tensión $V_{L_{off}}$ necesaria entre sus bornes para que siga circulando la corriente $I_{L_on}$. Sobre el MOSFET caen $V_{ds_{off}} = V_2 + V_D$ circulando la malla formada por el diodo, el MOSFET y la fuente de entrada. En este estado, si se desprecia la corriente parásita del MOSFET, toda la corriente $I_{L_{off}}$ de la bobina pasa por el diodo el cual se encuentra polarizado en directa a consecuencia de la tensión impuesta por la bobina y no se extrae corriente de la fuente de alimentación.

\begin{figure}[H]
	\centering
	\includegraphics[width=0.4\linewidth, page=2]{ImagenesEjercicio-1/CircuitsEj1}
	\caption{Hemicircuito con MOSFET encendido.}
	\label{ej1:fig:circuito_off}
\end{figure}

Resolviendo el circuito RL que se obtiene en este estado, se observa que la bobina tiene una pérdida de energía almacenada dada por las ecuaciones (\ref{ej1:eq:IL_off}) y (\ref{ej1:eq:VL_off}).

\begin{equation}
	 I_{L_{off}}(t) = \left( I_{L_{on}}\left(t= \frac{nD}{f_{sw}}\right) -\frac{V_d}{R_2}\right) e^{-\frac{R_2}{L}t} + \frac{V_d}{R_2} \ \ \ \ n \in \mathbb{N}_o
\label{ej1:eq:IL_off}
\end{equation}

\begin{equation}
	 V_{L_{off}}(t) = \left( V_d - I_{L_{on}}\left(t= \frac{nD}{f_{sw}}\right)\cdot R_2 \right)e^{-\frac{R_2}{L}t} \ \ \ \ n \in \mathbb{N}_o
\label{ej1:eq:VL_off}
\end{equation}

Donde $t=\frac{nD}{f_{sw}}$ son los momentos en los que el MOSFET conmuta de encendido a apagado y $I_{L_{on}}\left(t= \frac{nD}{f_{sw}}\right)$ la corriente en la bobina en dicho momentos. 

\subsubsection{Análisis en estado Permanente}
 Como la constante de tiempo $\frac{L}{R} = 189.6\mu s$ es una orden de magnitud mayor que el tiempo que se transcurre en cada estado $t_on = t_off = \frac{D}{f_{sw}} = \frac{0.5}{60kHz} = 8.333\mu s$ se pueden aproximar la tensión $V_L$ como constante y la corriente $I_L$ como rectas de pendiente $\frac{V_L}{L}$.

\subsubsection{Análisis de los Tiempos de Conmutación: Encendido}

Al encender el MOSFET con un escalón de tensión de $V_{GG} = 12V$, tomado en el borne izquierdo de $R_1$ y referido a masa, crece la corriente de gate $I_G$ instantáneamente a un valor de $I_G = \frac{V_i}{R_1} = 0.12A$ para luego decrecer exponencialmente según (\ref{ej1:eq:ig_ton}). Al mismo tiempo, la tensión entre gate y source pasa de ser nula a crecer exponencialmente según (\ref{ej1:eq:vgs_ton}).

\begin{equation}
I_G(t) = \frac{V_{GG}}{R_1}e^{-\frac{t}{\tau_1}}
\label{ej1:eq:ig_ton}
\end{equation}

\begin{equation}
V_{gs}(t) = V_{GG}(1-e^{-\frac{t}{\tau_1}})
\label{ej1:eq:vgs_ton}
\end{equation}

Donde la constante de tiempo $\tau_1 = R_1 (C_{gs}+C_{gd_{1}}) = 75ns$ es regida por la capacitancia de entrada del MOSFET compuesta por la capacitancia entre gate y source y la capacitancia entre gate y drain, con un valor de $C_{gs}+C_{gd_{1}} = 750pF$ observado en la Figura (\ref{ej1:fig:cgd}). Cuando la tensión entre gate y source llega al valor de threshold $V_{gs_{th}} = 5.5V$ (proporcionada por la datasheet del IRF530) luego de un tiempo 
\begin{equation}
t_{d_{on}} = -\tau_1 ln\left( 1-\frac{V_{th}}{V_2} \right) = 30.41ns
\label{ej1:eq:tdon}
\end{equation}

según la Ecuación (\ref{ej1:eq:vgs_ton}), comienza a crecer la corriente de drain $I_{ds}$ con pendiente constante. Una vez que la corriente de drain $I_{ds}$ alcanza el valor medio de la corriente de la bobina $I_L$, se observa en la Figura (\ref{ej1:fig:vgsio}) proporcionada por la datasheet del IRF530 que la tensión $V_{gs}$ será $V_{gs_{io}} = 5.5V$. Utilizando este valor y la Ecuación (\ref{ej1:eq:vgs_ton}) se obtiene que el tiempo de rise de la corriente de drain es

\begin{equation}
	t_{ri} = -\tau_1 ln\left( 1-\frac{V_{gs_{io}}}{V_2} \right) -t_{d_{on}} = 15.57ns
\label{ej1:eq:trise}
\end{equation}

A partir de este momento, se formará la zona de depleción de la body layer por lo que la tensión $V_{ds}$ comenzará a caer. La tensión $V_{gs} = V_{gs_{io}}$ y la corriente $I_G = \frac{V_{GG}-V_{gs_{io}}}{R_1} = 65mA$ se mantendrá constante mientras la capacitancia $C_{gd}$ se incrementa de $C_{gd_1}$ a $C_{gd_2}$ debido a la formación de la susodicha zona de depleción. Luego de un tiempo $t_{fv}$ en el que fluyó una carga de $\Delta Q = 7.3nC$ al gate del MOSFET, dato proporcionado de la datasheet del IRF530 visto en la Figura (\ref{ej1:fig:deltaq}), la tensión $V_{ds}$ alcanzará un valor $V_{ds} = V_{ds_{on}} = I_L R_{ds_{on}} = 827.59mV$ utilizando el valor de $I_L$ calculado en la Ecuación (). El tiempo transcurrido en esta transición puede calcularse teniendo en cuenta la corriente $I_G = \frac{V_2 - V_{gs_{io}}}{R_1}$ obteniendo

\begin{equation}
	t_{fv} = \frac{R_1 \Delta Q}{V_2 - V_{gs_{io}}} = 112.31ns
	\label{ej1:eq:tfv}
\end{equation}

Finalmente, habiendo transicionado la capacitancia $C_{gd}$ de $C_{gd_1}$ a $C_{gd_2}$, se terminará de cargar la capacitancia de entrada del MOSFET según las Ecuaciones (\ref{ej1:eq:vgs_ton_2}) y (\ref{ej1:eq:ig_ton_2}).

\begin{equation}
V_{gs}(t) = V_{GG}(1-e^{-\frac{t}{\tau_2}})
\label{ej1:eq:vgs_ton_2}
\end{equation}

\begin{equation}
I_{G}(t) =  \frac{V_{GG}-V_{gs_{io}}}{R_1}e^{-\frac{t}{\tau_2}}
\label{ej1:eq:ig_ton_2}
\end{equation}

siendo $\tau_2 = R_1(C_{gs} + C_{gd_2})$ donde $C_{gs} + C_{gd_2} = 1150pF$ observado en la Figura (\ref{ej1:fig:cgd}). Se puede observar en las Figuras (\ref{ej1:fig:encendido_gate}) y (\ref{ej1:fig:encendido_drain}) el proceso entero de encendido descrito anteriormente.

\begin{figure}[H]
	\centering
	\includegraphics[width=0.8\linewidth]{ImagenesEjercicio-1/encendido_gate}
	\caption{$V_{gs}$ e $I_G$ en el encendido del MOSFET.}
	\label{ej1:fig:encendido_gate}
\end{figure}

\begin{figure}[H]
	\centering
	\includegraphics[width=0.8\linewidth]{ImagenesEjercicio-1/encendido_drain}
	\caption{$V_{ds}$ e $I_D$ en el encendido del MOSFET.}
	\label{ej1:fig:encendido_drain}
\end{figure}

\begin{figure}[H]
	\centering
	\begin{minipage}{0.45\textwidth}
		\centering
		\includegraphics[width=\textwidth]{ImagenesEjercicio-1/Vgs-Id_LI} % first figure itself
		\caption{$V_{gs_{io}}$ de la datasheet del IRF530.}
		\label{ej1:fig:vgsio}
	\end{minipage}\hfill
	\begin{minipage}{0.45\textwidth}
		\centering
		\includegraphics[width=\textwidth]{ImagenesEjercicio-1/deltaq} % second figure itself
		\caption{$\Delta Q$ en la transición de $C_{gd_1}$ a $C_{gs_2}$ de la datasheet del IRF530.}
		\label{ej1:fig:deltaq}
	\end{minipage}
\end{figure}

\begin{figure}[H]
	\centering
	\includegraphics[width=0.4\linewidth]{ImagenesEjercicio-1/Vds-C}
	\caption{Capacitancia de entrada del MOSFET según la datasheet del IRF530 donde no ingresa carga a $C_{gs}$ debido a la tensión constante $V_{gs}$.}
	\label{ej1:fig:cgd}
\end{figure}

\subsubsection{Análisis de los Tiempos de Conmutación: Apagado}

\subsubsection{Corriente en el Diodo}

\subsubsection{Corriente en la Bobina}

\subsection{Circuito en la Simulación}

\subsection{Circuito en la Práctica}

\subsection{Diferencias}

\subsection{Conclusiones}

\end{document}