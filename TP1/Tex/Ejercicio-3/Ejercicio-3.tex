\input{../Informe/Header.tex}

\begin{document}

\subsection{Diferencias Switch Ideal - MOS}
En esta sección se reemplazará la llave ideal por un MOSFET con un circuito de disparo igual al del primer ejercicio.\\
Se observa en los gráficos 
\begin{multicols}{2}
\begin{figure}[H]
	\centering
	\includegraphics[width=0.9\linewidth]{ImagenesEjercicio-3/pend}
	\caption{Conmutaciones $V_{ds}$ e  $I_{ds}$.}
	\label{fig:ej3:conmutacionON_OFF_VDS_IDS}
\end{figure}
\begin{figure}[H]
	\centering
	\includegraphics[width=0.9\linewidth]{ImagenesEjercicio-3/pend}
	\caption{Conmutaciones $V_{gs}$ e  $I_{g}$.}
	\label{fig:ej3:conmutacionON_OFF_VGS_IG}
\end{figure}
\end{multicols}
\begin{multicols}{2}
\begin{figure}[H]
	\centering
	\includegraphics[width=0.9\linewidth]{ImagenesEjercicio-3/pend}
	\caption{Tensión y Corriente sobre la bobina.}
	\label{fig:ej3:Il_Vl}
\end{figure}
\begin{figure}[H]
	\centering
	\includegraphics[width=0.9\linewidth]{ImagenesEjercicio-3/pend}
	\caption{Tensión y Corriente sobre el diodo.}
	\label{fig:ej3:Id_Vd}
\end{figure}
\end{multicols}
Al realizar este cambio se puede notar cambios en muchas variables del circuito.
Se  se puede notar un cambio en la $V_L$ y que el duty cycle aumento respecto al que habia con una switch ideal, ademas que la tensión de salida ahora es $V_o \approx$
\note{chamullo del duty cycle mas grande y nueva tensión de salida}
También se puede observar que la corriente de reverse recovery del diodo ahora se ve acotada a $\approx 2.8[A]$, la cual es menor a la registrada en el caso anterior.

\subsection{Tiempos de Conmutación}
Los tiempos de conmutación se ven alterados respecto al circuito de la primera sección ya que los valores de $V_{gs-IO}$, $I_{g-IO}$ e $I_{ds}$ dependen principalmente del circuito de aplicación.
en este caso como en la topología Boost cuando el MOS se encuentra abierto se encuentra un circuito RLC mientras que cuando esta cerrado un RL del lado del generador y un RC en la carga esto afectará a los tiempos de $t_{ri}$ $t_{fv}$, $t_{d_{off}}$, $t_{rv}$ y  $t_{fi}$.\\
%\newpage

\begin{multicols}{2}
\begin{figure}[H]
	\centering
	\includegraphics[width=0.9\linewidth]{ImagenesEjercicio-3/pend}
	\caption{Conmutaciones $V_{ds}$ e  $I_{ds}$ llave con y sin Boost.}
	\label{fig:ej3:conmutacionON_OFF_VDS_IDS_SWITCH_BOOST}
\end{figure}
\begin{figure}[H]
	\centering
	\includegraphics[width=0.9\linewidth]{ImagenesEjercicio-3/pend}
	\caption{Conmutaciones $V_{gs}$ e  $I_{g}$ llave con y sin Boost.}
	\label{fig:ej3:conmutacionON_OFF_VGS_IG_SWITCH_BOOST}
\end{figure}
\end{multicols}
\begin{multicols}{2}
\begin{figure}[H]
	\centering
	\includegraphics[width=0.9\linewidth]{ImagenesEjercicio-3/pend}
	\caption{Tensión y Corriente sobre la bobina llave con y sin Boost.}
	\label{fig:ej3:Il_Vl_SWITCH_BOOST}
\end{figure}
\begin{figure}[H]
	\centering
	\includegraphics[width=0.9\linewidth]{ImagenesEjercicio-3/pend}
	\caption{Tensión y Corriente sobre el diodo llave con y sin Boost.}
	\label{fig:ej3:Id_Vd_SWITCH_BOOST}
\end{figure}
\end{multicols}

\end{document}