\documentclass[a4paper]{article}
\usepackage[utf8]{inputenc}
\usepackage[spanish, es-tabla, es-noshorthands]{babel}
\usepackage[table,xcdraw]{xcolor}
\usepackage[a4paper, footnotesep = 1cm, width=22cm, top=2.5cm, height=25cm, textwidth=20cm, textheight=25cm]{geometry}
%\geometry{showframe}

\usepackage{tikz}
\usepackage{amsmath}
\usepackage{amsfonts}
\usepackage{amssymb}
\usepackage{float}
\usepackage{graphicx}
\usepackage{caption}
\usepackage{subcaption}
\usepackage{multicol}
\usepackage{multirow}
\usepackage{wrapfig}
\setlength{\doublerulesep}{\arrayrulewidth}
\usepackage{booktabs}

\usepackage{hyperref}
\hypersetup{
    colorlinks=true,
    linkcolor=blue,
    filecolor=magenta,      
    urlcolor=blue,
    citecolor=blue,    
}

\newcommand{\note}[1]{
	\begin{center}
		\huge{ \textcolor{red}{#1} }
	\end{center}
}

\setcounter{topnumber}{2}
\setcounter{bottomnumber}{2}
\setcounter{totalnumber}{4}
\renewcommand{\topfraction}{0.85}
\renewcommand{\bottomfraction}{0.85}
\renewcommand{\textfraction}{0.15}
\renewcommand{\floatpagefraction}{0.8}
\renewcommand{\textfraction}{0.1}
\setlength{\floatsep}{5pt plus 2pt minus 2pt}
\setlength{\textfloatsep}{5pt plus 2pt minus 2pt}
\setlength{\intextsep}{5pt plus 2pt minus 2pt}

\newcommand{\quotes}[1]{``#1''}
\usepackage{array}
\newcolumntype{C}[1]{>{\centering\let\newline\\\arraybackslash\hspace{0pt}}m{#1}}
\usepackage[american]{circuitikz}
\usetikzlibrary{calc}
\usepackage{fancyhdr}
\usepackage{units} 

\graphicspath{{../Ejercicio-1/}{../Ejercicio-2/}{../Ejercicio-3/}{../Ejercicio-4/}{../ParteI/}{../ParteII/}{../ParteIII/}{../ParteIV/}}

\pagestyle{fancy}
\fancyhf{}
\lhead{22.14 - Electrónica IV}
\rhead{Mechoulam, Lambertucci, Londero}
\rfoot{Página \thepage}


\begin{document}

\subsection{Diferencias Switch Ideal - MOS}
En esta sección se reemplazará la llave ideal por un MOSFET con un circuito de disparo igual al del primer ejercicio.\\
Al realizar este cambio se puede notar cambios en muchas variables del circuito.
Se  se puede notar un cambio en la $V_L$ y que el duty cycle aumento respecto al que habia con una switch ideal, ademas que la tensión de salida ahora es $V_o \approx$
\note{chamullo del duty cycle mas grande y nueva tensión de salida}
También se puede observar que la corriente de reverse recovery del diodo ahora se ve acotada a \note{VALOR DEL IRR}, la cual es menor a la registrada en el caso anterior.

\note{foto de la corriente de bobina y MOS en conmutacion llave ideal y mos}
\subsection{Tiempos de Conmutación}
Los tiempos de conmutación se ven alterados respecto al circuito de la primera sección ya que los valores de $V_{gs-IO}$, $I_{g-IO}$ e $I_{ds}$ dependen principalmente del circuito de aplicación.
en este caso como en la topología Boost cuando el MOS se encuentra abierto se encuentra un circuito RLC mientras que cuando esta cerrado un RL del lado del generador y un RC en la carga.

\note{foto con tiempos de conmutación de llave sin boost y con boost}
\end{document}