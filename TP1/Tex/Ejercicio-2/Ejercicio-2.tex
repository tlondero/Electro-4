\input{../Informe/Header.tex}

\begin{document}

\subsection{Introducción}

Dada una fuente Boost con una tensión de entrada $12 \ V$ y frecuencia de switching de $60 \ kHz$, se buscó determinar el Duty Cicle necesario tal que la tensión de salida sea de $24 \ V$ y tenga una variación del $5\%$. Cabe notar que esta fuente Boost es una no ideal ya que se considera la resistencia de la bobina $R_4 = 2 \ \Omega$.

\begin{figure}[H]
	\centering
	\includegraphics[width=0.7\linewidth, page=1]{ImagenesEjercicio-2/CircuitsEj2}
	\caption{Circuito de fuente Boost con llave ideal.}
	\label{fig:ej2:circuito}
\end{figure}

\subsection{Calculo del Duty Cicle}

Para el período de encendido, el hemicircuito es el siguiente. 

\begin{figure}[H]
	\centering
	\includegraphics[width=0.8\linewidth, page=2]{ImagenesEjercicio-2/CircuitsEj2}
	\caption{Circuito de fuente Boost con llave cerrada.}
	\label{fig:ej2:off}
\end{figure}

Planteando las mallas 1 y 2 y operando algebraicamente, se obtienen lo coeficientes de la matriz $\mathbb{A}_{on}$.
\begin{equation}
\mathbb{A}_{on} =  \begin{bmatrix}
	-\frac{R_4}{L} & 0 \\
	0 & -\frac{1}{C R_2}
\end{bmatrix}
\end{equation}

Por otro lado, durante el apagado, el hemicircuito resultante es el que se muestra a continuación.

\begin{figure}[H]
	\centering
	\includegraphics[width=0.8\linewidth, page=3]{ImagenesEjercicio-2/CircuitsEj2}
	\caption{Circuito de fuente Boost con llave cerrada.}
	\label{fig:ej2:on}
\end{figure}

Para obtener $\mathbb{A}_{off}$ bastan con plantear la malla externa, la malla 2 planteada previamente (ya que no ha cambiado) y la suma de las corrientes $I_L$, $I_C$ e $I_O$.

\begin{equation}
\mathbb{A}_{off} =  \begin{bmatrix}
	-\frac{R_4}{L} & -\frac{1}{L} \\
	\frac{1}{C} & -\frac{1}{C R_2}
\end{bmatrix}
\end{equation}

Para ambos casos se obtiene que las matrices $B$ y $C$ son las mismas, siendo estas:
\begin{equation}
\mathbb{B} =  \begin{bmatrix}
	\frac{1}{L} \\
	0
\end{bmatrix}
\end{equation}

\begin{equation}
\mathbb{C} =  \begin{bmatrix}
	0 & 1
\end{bmatrix}
\end{equation}

Finalmente, definiendo la matriz $\mathbb{A}$ de la forma $\mathbb{A} = \mathbb{A}_{on} \cdot d + \mathbb{A}_{off} \cdot \left( 1 - d \right)$, se obtiene que la transferencia en el permanente esta dada por la siguiente formula:
\begin{equation}
H = -\mathbb{C} \cdot \mathbb{A}^{-1} \cdot \mathbb{B} = \frac{\left( 1 - d \right) R_2}{R_2 d^2 + 2 d \left( R_4 - R_2 \right) + R_2 - R_4}
\end{equation}

Reemplazando con $R_4 = 2 \ \Omega$, $R_2 = 100 \ \Omega$, y sabiendo que se busca que $H = V_o / V_2 = 2$, se obtiene que el Duty Cicle puede tomar 2 valores posibles: $d = 0.5$ y $d = 0.96$.

%Utilizando la transferencia de la fuente Boost ideal, es decir sin $R_4$, se puede obtener el Duty Cicle deseado:
%
%\begin{align*}
%V_o &= \frac{V_2}{1 - d}	\\
%1 - d &= \frac{V_2}{V_o} = \frac{12 \ V}{24 \ V} \\
%d &= 0.5
%\end{align*}

\end{document}