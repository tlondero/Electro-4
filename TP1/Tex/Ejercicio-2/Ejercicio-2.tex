\input{../Informe/Header.tex}

\begin{document}

\subsection{Introducción}

Dada una fuente Boost con una tensión de entrada $12 \ V$ y frecuencia de switching de $60 \ kHz$, se buscó determinar el Duty Cicle necesario tal que la tensión de salida sea de $24 \ V$ y tenga una variación del $5\%$. Cabe notar que esta fuente Boost es una no ideal ya que se considera la resistencia de la bobina $R_4 = 2 \ \Omega$.

\begin{figure}[H]
	\centering
	\includegraphics[width=0.7\linewidth, page=1]{ImagenesEjercicio-2/CircuitsEj2}
	\caption{Circuito de fuente Boost con llave ideal.}
	\label{fig:ej2:circuito}
\end{figure}

\subsection{Calculo del Duty Cicle}

%Utilizando la transferencia de la fuente Boost ideal, es decir sin $R_4$, se puede obtener el Duty Cicle deseado:
%
%\begin{align*}
%V_o &= \frac{V_2}{1 - d}	\\
%1 - d &= \frac{V_2}{V_o} = \frac{12 \ V}{24 \ V} \\
%d &= 0.5
%\end{align*}

\end{document}