\documentclass[border={0.5cm 0.5cm 0cm 0.5cm}, 11pt, tikz, multi=page]{standalone}
\usepackage[utf8]{inputenc}
\usepackage[spanish, es-tabla, es-noshorthands]{babel}

\usepackage{tikz}
\usepackage{textcomp}
\usetikzlibrary{shapes,arrows}

\usepackage{amsmath}
\usepackage{amsfonts}
\usepackage{amssymb}
\usepackage{float}
\usepackage{graphicx}
\usepackage{caption}
\usepackage{subcaption}
\usepackage{multicol}
\usepackage{multirow}
\setlength{\doublerulesep}{\arrayrulewidth}
\usepackage{booktabs}
\usepackage{pgfplots}

\newcommand{\quotes}[1]{``#1''}
\usepackage{array}
\newcolumntype{C}[1]{>{\centering\let\newline\\\arraybackslash\hspace{0pt}}m{#1}}
\usepackage[american]{circuitikz}
\usepackage{fancyhdr}
\usepackage{units}

% Definition of blocks:
\tikzset{%
  block/.style    = {draw, thick, rectangle, minimum height = 3em,
    minimum width = 3em},
  sum/.style      = {draw, circle, node distance = 2cm}, % Adder
  input/.style    = {coordinate}, % Input
  output/.style   = {coordinate}, % Output
  >=Stealth
}

% Defining string as labels of certain blocks.
\newcommand{\suma}{\Large $\Sigma$}
\newcommand{\inte}{$\displaystyle \int$}
\newcommand{\derv}{\huge $\frac{d}{dt}$}
\usepackage{siunitx}
\begin{document}

%%%%%%%%%%%%%%%%%%%%%%%%%%%%%%%%%%%%%%%%%%%%%%%%%%%%%%%%%%%%%%%%%%%%%
%							COMPONENTES								%
%%%%%%%%%%%%%%%%%%%%%%%%%%%%%%%%%%%%%%%%%%%%%%%%%%%%%%%%%%%%%%%%%%%%%

%Voltage controlled switch
\newcommand{\switch}[1] % #1 = name
{node(#1_origin){}
	node[dipchip, num pins=6, hide numbers, no topmark, external pins width=0](C3){}
	
	%($ (C3.bpin 1) !.5! (C3.bpin 4) $) ++ (0,0.5) node[](){}

	%LEFT NODES
	node[right, font=\footnotesize] at (C3.bpin 1) {$+$}
	node[right, font=\footnotesize] at (C3.bpin 3) {$-$}		

	(C3.bpin 1) -- ++ (-0.5,0) node[](#1_plus){}	
	(C3.bpin 3) -- ++ (-0.5,0) node[](#1_minus){}
	

	%RIGHT NODES
	node[left, font=\footnotesize] at (C3.bpin 4) {}	
	node[left, font=\footnotesize] at (C3.bpin 6) {}
	
	(C3.bpin 6) -- ++(-0.5,0) node[](xd){}
	(C3.bpin 4) -- ++(-0.5,0) to[nos] (xd.center)
	
	(C3.bpin 4) -- ++ (0.5,0) node[](#1_s2){}
	(C3.bpin 6) -- ++ (0.5,0) node[](#1_s1){}			
}

%%%%%%%%%%%%%%%%%%%%%%%%%%%%%%%%%%%%%%%%%%%%%%%%%%%%%%%%%%%%%%%%%%%%%
%							CIRCUITOS								%
%%%%%%%%%%%%%%%%%%%%%%%%%%%%%%%%%%%%%%%%%%%%%%%%%%%%%%%%%%%%%%%%%%%%%

%CIRCUITO COMPLETO EJERCICIO 2
\begin{page}
\begin{circuitikz}[american voltages]

	\def\xspace{0.75}
	
	\ctikzset{multipoles/thickness=2}				%Ancho de las lineas de las cajas
	\ctikzset{diodes/scale=0.6}	
	\ctikzset{multipoles/dipchip/width=1}			%Ancho del swirch
		
	\draw
		(0,0) \switch{sw}
		(sw_minus.center) -- ++ (0,-1) node[ground](){}
		(sw_s2.center) -- ++ (0,-1) node[ground](){}
		(sw_plus.center) -- ++ (-1,0) to[V, l_=$V_1$] ++ (0,-2.1) node[ground](){}
		
		(sw_s1.center) -- ++ (0,1) node[](aux1){} to[D, l_=$\begin{array}{c} D_1 \\ MUR460\end{array}$] ++ (3,0) node[](aux2){} to[C, l=$\begin{array}{c} C_1 \\ \SI{10}{\mu f}\end{array}$] ++ (0,-3.25) node[ground](){}
		(aux2.center) -- ++ (2,0) to[R, l=$\begin{array}{c} R_2 \\ \SI{100}{\ohm}\end{array}$] ++ (0,-3.25) node[ground](){}
		
		(aux1.center) to[R, l=$\begin{array}{c} R_4 \\ \SI{2}{\ohm}\end{array}$] ++ (0,2) to[L, l=$\begin{array}{c} L_1 \\ \SI{220}{\mu H}\end{array}$] ++ (0,2) -- ++ (-5,0) to[V, l=$\begin{array}{c} V_2 \\ \SI{12}{V}\end{array}$] ++ (0,-7.1) node[ground](){}		
	;
	
%	\draw[color = red]
%		(aux1.center) node[circ](){1}
%		(aux2.center) node[circ](){2}
%	;

\end{circuitikz}
\end{page}

%HEMICIRCUITO OFF
\begin{page}
\begin{circuitikz}[american voltages]

	\node[ground](){};	
	\ctikzset{diodes/scale=0.6}
	\draw	
	
		(0,0) node[ground](){} to[R, l_=$\begin{array}{c} R_4 \\ \SI{2}{\ohm}\end{array}$] ++ (0,1.5) to[L, l_=$\begin{array}{c} L_1 \\ \SI{220}{\mu H}\end{array}$, v^<=$V_L$, i_<=$I_L$] ++ (0,2) -- ++ (-3,0) to[V, l_=$\begin{array}{c} V_2 \\ \SI{12}{V}\end{array}$] ++ (0,-3.5) node[ground](){}
		
		(3,0) node[ground](){} to[C, l=$\begin{array}{c} C_1 \\ \SI{10}{\mu f}\end{array}$, i^<=$I_C$, v_<=$V_C$] ++ (0,3.5) to[short, i>^=$I_O$] ++ (3,0) to[R, l=$\begin{array}{c} R_2 \\ \SI{100}{\ohm}\end{array}$, v_>=$V_O$] ++ (0,-3.5) node[ground](){}	
		;
		
	%MALLA 1
	\draw[<-,shift={(-1.5,0.9)},color=red] (0:.7cm) arc (0:180:.7cm) node at(0,0){};
	\draw[->,shift={(-1.5,0.9)},color=red] (0:.7cm) arc (0:-180:.7cm) node at(0,0){};
	\draw[color=red] (-1.5,0.5) node[label=1](){};
	
	%MALLA 2
	\draw[->,shift={(4.5,0.9)},color=red] (0:.7cm) arc (0:180:.7cm) node at(0,0){};
	\draw[<-,shift={(4.5,0.9)},color=red] (0:.7cm) arc (0:-180:.7cm) node at(0,0){};
	\draw[color=red] (4.5,0.5) node[label=2](){};

\end{circuitikz}
\end{page}

%HEMICIRCUITO ON
\begin{page}
\begin{circuitikz}[american voltages]

	\ctikzset{diodes/scale=0.6}
	\draw	
	
		(0,0) to[R, l=$\begin{array}{c} R_4 \\ \SI{2}{\ohm} \\ \\ \end{array}$] ++ (2,0) to[L, l_=$\begin{array}{c} L_1 \\ \SI{220}{\mu H} \\ \\ \end{array}$, i_>=$I_L$, v^>=$V_L$] ++ (4,0) node[](aux1){} to[C, l=$\begin{array}{c} C_1 \\ \SI{10}{\mu f}\end{array}$, i>^=$I_C$, v_>=$V_C$] ++ (0,-2) node[ground](){}		
		(aux1.center) to[short, i=$I_O$] ++(1,0) -- ++ (2,0) to[R, l=$\begin{array}{c} R_2 \\ \SI{100}{\ohm}\end{array}$, v_>=$V_O$] ++ (0,-2) node[ground](){}
		(0,0) to[V, l_=$\begin{array}{c} V_2 \\ \SI{12}{V}\end{array}$] ++ (0,-2) node[ground](){}	
		(aux1.center) node[circ, label=above:$A$](){}
		;
	
%	%MALLA 3
%	\draw[<-,shift={(4.5,0.75)},color=red] (0:.7cm) arc (20:160:.7cm) node at(0,0){};
%	\draw[color=red] (4.5,0.35) node[label=3](){};
%	\draw[european voltages, color=red] (0,1.25) to[open, v^>=$3$] ++ (10,0);
		
%	\draw[color = red]
%		(aux1.center) node[circ](){1}
%	;

\end{circuitikz}
\end{page}

\end{document}